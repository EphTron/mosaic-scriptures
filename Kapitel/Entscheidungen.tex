\MosaicChapter{German}{Entscheidungen}{Ein Paralleluniversum, eine Galaxis, einen Kontinent, ein Zeitalter, eine Gedankenwelt, aber vielleicht auch nur einen Brief entfernt:}{Ephraim Schott, Amaya Gallegos}%\Chapter{Entscheidungen}
\Authors{Ephraim Schott, Amaya Gallegos}
% \lettrine{}{} command typesets a drop cap in the document

% second brace set formats text placed inside as small caps
% (same thing as doing \lettrine{L}\textsc{orem ipsum ...}

\lettrine{O}{hne groß darüber nachzudenken}, trat ich in den dunklen Höhleneingang. Der Gang war schmal und wurde immer schmaler. Ich musste den Kopf leicht einziehen, um ihn mir nicht zu stoßen und nach ein paar Schritten wurde mir klar, dass ich Platzangst bekam und ich beschleunigte meine Schritte. Der viel zu schwache Schein meiner Handytaschenlampe glitt an ein paar Spinnweben entlang. Ich schauderte. Die Höhlenwände waren nass und es tropfte von der Decke herunter. Ein paar Tropfen vielen mir auf den Kopf. Ich ging noch schneller, rannte nun fast. Da vernahm ich plötzlich ein leises Kreischen. Das müssen Fledermäuse sein, dachte ich gar nicht erpicht darauf, meiner Vermutung zu bestätigen. Wann komme ich endlich wieder zum Ausgang, der Guide sagte doch, dass es hier einen zweiten Ausgang gäbe? Doch stattdessen gelangte ich nun an eine Abzweigung. Vor mir zwei gähnende schwarze Schlünde, der gleichen Größe. Durch keinen der beiden schimmerte Licht. Ich schnupperte. Keiner der beiden, roch nach frischer Luft. Was mache ich jetzt? Recht, links oder zurück?

Aus Gewohnheit blickte ich auf meinen Handybildschirm. 57\% Akku. Kein Empfang.
Wie viele dieser Entscheidungen hatte ich in meinem Leben bereits getroffen? Nudeln, Kartoffeln oder Reis war es gestern Abend gewesen. Ich drehte mich um und leuchtete in den schmalen Gang aus dem ich kam. Kein beruhigender Lichtpunkt, der auf den Eingang hindeutete, war mehr zu sehen. A den nackten, kalten Wänden lief stellenweise Wasser herunter und sammelte sich in kleinen Pfützen zwischen den Steinen am Boden. Heute morgen war die Entscheidung helle oder dunkle Hose gewesen. Ich leuchtete kurz an mir runter und die dunklen Flecken auf meiner hellen Hose wiesen mich darauf hin, dass ich definitiv die falsche Wahl getroffen hatte. Trottel. Aber egal, ich war ja allein und hatte jetzt ja andere Sorgen. Ich drehte mich wieder der Gabelung zu. Links, Rechts oder zurück? 56\%. Diese Entscheidung schien jetzt bedeutender als Reis oder Nudeln. Ich musste schmunzeln. Das war auch der Grund gewesen warum ich mir die Höhlentour ausgesucht hatte: Raus aus dem Alltag. Zurück war also keine Option. Ich kannte alle belanglosen Abzweigungen die hinter mir lagen. Ich war ja hier um was neues zu erleben. Also, rechts oder links?

