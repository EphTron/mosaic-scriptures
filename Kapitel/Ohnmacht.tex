\MosaicChapter{German}{Ohnmacht}{Ein Paralleluniversum, eine Galaxis, einen Kontinent, ein Zeitalter, eine Gedankenwelt, aber vielleicht auch nur einen Brief entfernt:}{Ephraim Schott, Johannes Hartmann}
%\Chapter{Ohnmacht}
\Authors{Ephraim Schott, Johannes Hartmann}
% \lettrine{}{} command typesets a drop cap in the document

% second brace set formats text placed inside as small caps
% (same thing as doing \lettrine{L}\textsc{orem ipsum ...}

\lettrine{D}{ie, grauen Vorhänge schimmerten bunt durch} das hereinfallende Licht der grellen Werbeanzeigen und neon Leuchtschriften der Bars und Lounges, die Soho in ein rastloses Farbenspiel tauchten. Schatten von vorbeischwirrenden Drohnen tanzten ununterbrochen über die zugezogenen Vorhänge des kleinen Apartments. Harald drückte eine selbstgedrehte Zigarette in seinem Aschenbecher aus. Eines Tages würde ihn diese Sucht sein Leben kosten, daran hatte er keine Zweifel.
“Heute war es besonders schlimm. Ich weiß nicht wie lange ich diesen Job noch aushalte und die aktuellen Nachrichten machen es auch nicht gerade besser.” sagte Amir, zuckte mit den Schultern und zwang sich zu einem müden Lächeln. “Wie machst du das? Wie gehst du mit der ganzen scheiße um?”
Harald’s Blick wanderte zu seinem Freund der auf der Couch saß und weiter zu dem Schuhkarton, den er hastig unter die Couch geschoben hatte, als es an der Tür geklingelt hatte - wenn Amir wüsste. Kurz überlegt Harald ob er seinen alten Freund einweihen sollte, doch jetzt hatte er es schon soweit gebracht. Alles war vorbereitet. Morgen würde er den CEO der größten englischen Housing Corporation erschießen. Es war ein Ein-Mann-Job. Er hatte den Plan alleine geschmiedet und hatte nicht vor unnötig andere Menschen, gar Freunde, mit hineinzuziehen.
“…Harald? Hörst du mir eigentlich zu? Ich hab dich was gefragt!” Riss ihn Amir aus seinen Gedanken und schaut ihn fragend an. “Wie gehst du mit dieser Ohnmacht um?”

Harald zuckte zusammen, aufgrissen aus seinen Gedanken, leicht zornig, leicht verschreckt, in jedem fall fiel seine Antwort impulsiver aus, als er es sich gewünscht hätte;
``Mein Gott, was weiß ich Amir, welch scheiß Ohnmacht?''
Harald gesitkulierte auf die ausgedrückte Zigratette.
``Welche scheiß Ohnmacht, Junge? Wenn ich hier und jetzt entscheide, dass das die letzte Zigratette ist die ich jemals rauche, und das durchziehe, dann meine Fresse, hat die Zigarette keine Macht mehr über mich. Wenn ich hier und jetzt entscheide aus dem verschissenen siebten Stock zu springen, dann hat niemadn je wieder irgendeine Macht über mich, weißt du? Es gibt keine Ohnmacht. Es gibt nur Lethargie, Lethargie zu reden, zu handeln, zu tun. Lethargie sich zu wehren, weißt du? Es gibt kein... es gibt gar nichts! Ohnmacht ist doch auch nur irgendwas, was die ganzen verschissenen systemtreuen Medien uns erzählen, ja? Die ganzen, die ganzen Medien die sich als alternativ'' - und er betonte ``alternativ'' auf eine Art und Weiße die irgendwo zwischen Wehmut und Mitleid lag - ``schimpfen. Sind genauso gekauft. Genauso geschmiert. Genauso gefickt. Schreiben halt von Ohnmacht, weißt du? Damit wir, wir, die wenigen die vielleicht noch genug wut hätten rauszugehen und was zu machen, auch ja das Narrativ fressen. Ohnmacht dies, Ohnmacht das. Wehrlos dies, wehrlos das. Too big to fail! Too big to fail! Das haben sie gesagt, über die scheiß Römer, und die sind alle tot. Über die scheiß Briten, und die sind alle tot. Über die Scheiß USA, und die sind alle tot. Und weißt du warum? Weil irgendwer einen Fick gegeben hat auf Ohnmacht, und einfach gehandelt hat.''

Für einen Moment hatte Harald Sorge, dass Amir den Braten riechen würde, doch dann wurde klar, dass dem nicht so war. Amir blickte etwas verduzt in Richtung seines Freundes. ``Wow, Harald, man, du bist ja noch wütender als sonst. Gehts dir gut? Sollten wir noch ein' trinken, man, damit du bisschen runterkommst?''

Vielleicht war das keine schlechte Idee, dachte sich Harald. Hatte er überreagiert? Amir war einer seiner zarter besaiteteren Freunde. Wie er ihn so auf dem Sofa sitzen sah, konnte er sich noch gut zurückerinnern, als er den arabischstämmigen Mann damals auf der Semesterparty der Offenen Universität London erblickte hatte. Er hatte alleine und verloren an der Bar gestanden und Harald hatte ihn mit in seine Gruppe aufgenommen. Sie hatten Pillen geschmissen und bis in die Morgenstunden gefeiert und sich über das Leben unterhalten. Schon früh fühlten sie sich beide durch ihre ähnliche Weltsicht verbunden und nach dem Ende ihres Informatikstudiums blieben sie in regelmäßigen Kontakt. 

``Ja, lass uns noch einen trinken!'' nickte Harald zu Amir und machte Anstalten aufzustehen.
``Bleib sitzen'' entgegnete Amir und verschwand in der Küche.
Harald griff zum Tabak und began sich eine neue Zigarette zu drehen. Das war so eine Sache mit den Entscheidungen und der Lethargie. Harald´s Ausbruch brachte ihn zum grübeln.
Er war eigentlich überzeugt gewesen, dass ihn die Vorbereitungen der letzten Monate und das näherrückende Ziel nicht beeinflussten. Natürlich gab er immer gerne seine Meinung und auch seine Wut zur immer schlimmer werdenden politischen Situation zum besten, aber jeder der Harald kannte wusste, dass er nicht zu wilden Ausbrüchen neigte. Stress merkte man ihm grundsätzlich nur selten an. Viele hätten ihn vermutlich eher mit dem Wort ``kontrolliert'' beschrieben. Er war von sich selbst überrascht. Er hatte das gesamte Jahr auf einen Tag, seinen ``Stichtag'', hingearbeitet. Er war dabei so methodisch vorgegangen, dass er seine Wut manchmal ganz vergas. Es waren Momente wie gerade eben, die ihm zeigten, dass die Emotionen noch da waren.

Im Laufe der letzten Monate war er tagsüber seiner Arbeit bei den Global Railway Services nachgegangen und hatte die Nächte damit zugebracht einen Plan zu schmieden, der ihm wieder eine Stimme geben sollte. Sein Zeit als Verstummter war ab morgen vorbei und sollte nicht nur ein einziger verzweifelter Aufschrei werden. Harald hatte nicht vor sich seine Stimme wieder nehmen zu lassen. Anfangs, war er überrascht gewesen, wie einfach es war sich eine Pistole 3D zu Drucken, doch schnell verstand er, dass es das Spuren verwischen war was seinen Plan kompliziert machte. ``Vor einzelnen Aktionen haben die keine Angst, es ist die Macht es gezielt wiederholen zu können, vor der sie sich die da oben wirklich fürchten” murmelte Harald leise vor sich hin. Fuck! Schon wieder?

``Was fürchten die da oben?'' fragte Amir scherzend und kam aus der angrenzenden Küche mit zwei Gläsern goldenem Whiskey zurück ins Wohnzimmer. ``Hier Harald. Ich glaub du brauchst einen'' sagte er mit einem lächeln und stellte eines der bauchigen Gläser knallend vor Harald ab. Er steckte sich eine Hand in die Hosentasche und schlenderte zu den verhängten Panoramafenstern. Er zog einen der Vorhänge auf, stellte sich an das große Fenster und blickte hinaus auf die schillernde Skyline Londons. Eine Drohne schwirrte am Fenster vorbei. Harald zündete sich seine Zigarette an ``Sorry wegen eben.''

``Ach, weißt du, es ist nicht so dass ich dich nicht verstehe. Klar, ich kann mich hier und jetzt aus dem Fenster stürzen, aber das macht nichts besser. Nicht für mich, nicht für dich, und dem System, denen da oben ist es egal. Denen ist es auch egal wenn hundertfünfzig Arbeiter verrecken weil in irgendeinem ihrer Chemkraftwerke ein Tank hochgeht. Merken die ja gar nicht. Aber was will man tun?'' sagte Amir, ohne sich Harald wieder zuzudrehen. Er folgte mit seinem Blick den Drohnen die vorbeiflogen.

``Irgendwas. Aufbegeheren? Agitieren?''
``Bringt doch nichts. Wenn du auch nur anfangen würdest würden sie dich sofort aus dem System ziehen. Kein Tritt bleibt unbemerkt. Keine Suchanfrage kann nicht auf dich zurückgeführt werden. Glaubst du wirklich dass nicht jede Drohne die hier vorbeifliegt in der Lage ist unser Gespräch abzuhören?''

``Meinst du das können die?'' fragte Harald und zog mit gerunzelter Stirn an der Zigarette.

``Bruuuder, die können noch viel mehr! Ich wette die Hälfte der Dinger ist ausgestattet dich sofort auszuschalten wenn du die falschen Worte sagst.''

Harald hielt einen Moment inne. War das wirklich so? Nein, mit Amir ging die Fantasie durch... oder? 
``Ich glaube du bist ein bisschen paranoid, mein Guter.''

``Ich bin paranoid, klar man, muss man doch. Sag ein falschen Satz gegen das System und die Schatten fangen an sich zu schlängeln. Dieses ganze System funktioniert doch nur weil dafür gesorgt wird dass die, die sich aufbegehren wollen, sich nicht vernetzen können. Häng ein Flugblatt auf, zack, weggesperrt. Verfassungsfeind gegen die Kapitalistische Demokratie. Mach einen Schritt ins falsche Gebäude, zack, umgenietet. Warst Extremist. Klare Kiste. Wie, es gibt keine Beweise? Brauchen wir nicht, wir haben hier reihenweiße Dokumente die belegen, dass du dich radiaklisiert hast. Wo wir die herhaben fragt dann keiner mehr. Wir sind unterdrückt, und dieses System der neoliberalen, neokapitalisitschen Demokratie hat es geschafft ein Netz zu formen das alles legitimiert, wenn man nur behaupten kann, dass man Wohlstand und Besitz schützen wollte. Stell das nicht in Frage, oder wir kriegen dich. Du Ratte, du willst doch Wohlstand aus dem System ziehen!'' Amir schnaubte. Amir schüttelte sich, und schlug die Hände über dem Kopf zusammen. ``Wir sind wehrlos. Es ist hoffnungslos.''

``Aber mir erzählst du, dass ich wütend bin? Ich glaube du brauchst einen Drink, komm mal rüber hier und baller dir ein' hinter den Denkdeckel meiner!''

``Ja, vermutlich hast du recht. Erstmal ein' reinsaufen, dann sieht die Welt schon anders aus.''

Beide nahmen einen großen Schluck aus ihrem Drink, dann nickte Amir. ``Weißt du, ich hab nur - ich hab wieder dieses verschissene Video von Erin Bloughtborrough gesehen, diesem Hurensohn von Vonovia Fort Europe, dieser Wichser-CEO, auf seiner Wichser-Yacht, und wie er die gewaltsame Niederschlagung der Proteste gegen sein neues Wichserprojekt verteidigt hat von einem Jahr. `Wir als Vonovia Fort Europe wir werden jedes Mittel nutzen um jene, die der Ansicht sind dass sie anderen ihren wohl verdienten Wohlstand entziehen können, daran zu hindern', wäwäwä, was für ein Arschgesicht!''

Harald zuckte. Bloughtborrough. Über diesen Mann würde es morgen früh viele Videos geben. Aber wenn alles nach Plan lief würde Bloughtborrough in diesen Videos nichts mehr von sich geben. Harald hatte das Video auch gesehen, ohne jede Frage war dieses Video auch Teil seiner Entscheidung gewesen - hatte ihm gewissermaßen Inspiration gegeben. Jedes Mittel nutzen. Harald würde es ihm gleich tun.

``Ich verstehe dass Bloughtborrough dich wütend macht. Sehr sogar.'' Harald seuzfte.

Er steckte sich die Kippe in den Mundwinkel, schnappte sich sein Glas und trat neben Amir ans Fenster. Tief unter ihnen in den Straßenschluchten schoben sich Autos durch den niemals endenden Verkehr.

Vor dem großen Börsencrash wäre eine Wohnung wie diese unbezahlbar gewesen, doch als innerhalb von nur einem Monat fast die gesamte Mittelschicht des Westens in die Armut abrutschte, reagierten die Giga Corporations schnell und stellten ihre Miets- und Geschäftsmodelle um. Das neue System nannte sich Benefits Grid und ersetzt Löhne mit Subscriptions und Abos. Ob Wohnung, Fitnessstudio, Essen oder Videoportal, alles wurde von Konzernen bereitgestellt, wenn man nur tüchtig anpackte. Da in nur wenigen Tagen alle Verträge angepasst waren, munkelte manch einer, dass der Crash nur ein Vorwand für die Umstellung gewesen war. In der Flut an Nachrichten die in dieser Zeit um die Welt ging, ließ sich jedoch keine eindeutige Wahrheit mehr erkennen - jeder durfte glauben was er wollte.

“Weißt du was mir noch Hoffnung macht Amir? Die ganzen Nachrichtenbots und Mainstream News, ja selbst die Alternativen Medien, erzählen alle irgendwelche Geschichten, aber eigentlich interessieren die schon lange niemanden mehr. Klar, jeder schaltet noch ab und zu seinen Lieblingskanal an, um ein letztes Gefühl von Kontrolle zu wahren, aber alles nur noch aus Gewohnheit. Tief im Inneren weiß doch jeder, dass alles eine Farce ist. Keiner glaubt den Scheiß mehr! Und noch was: Keiner mag mehr an Fairness glauben...”Harald machte eine kunstvolle Pause und nahm die Kippe aus seinem Mundwinkel, um demonstrative daran zu ziehen
“..., aber jeder weiß noch was fair ist! Der Fairness-Kompass funktioniert noch.” Harald lächelte verschwörerisch und nippte genüßlich an seinem Whiskey. 
“Klar jeder kriegt andere Lügen aufgetischt und hat jemand den er für Schuldig erklären kann; das Ziel der Medienalgorithmen ist ja leider immer noch “Divide and Conquer”, aber solange die Nadel noch reagiert, hab ich Hoffnung. Die Medien senden ein massives Störsignal, aber grundsätzlich funktioniert der Fairness-Kompass noch! Jeder verspürt noch Genugtuung, wenn jemand von der anderen Gruppe bestraft wird. Jeder hofft, dass sich das Blatt zu seinen Gunsten wendet. Nur leider ist die Informationslage echt beschissen.
…
 Niemand sympathisiert mehr mit denen da oben.”

Amir sagte nichts und tat es Harald gleich, nur dass er dabei nicht lächelte. Seine Stirn war von tiefen Sorgenfalten durchzogen. “AMK! Als wir angefangen haben zu studieren, da hatte ich wirklich high hopes. Ich dachte wirklich ich könnte einen Unterschied machen” began Amir grimmig.
“Mir war schon bewusst”  
Ich hatte eine gute Vision für die Welt, aber heute gibt es nur noch die Wahl zwischen beschissen und beschissener.



%Er steckte sich die Kippe in den Mundwinkel, schnappte sich sein Glas und trat neben Amir ans Fenster. Tief unter ihnen in den Straßenschluchten schoben sich Autos durch den niemals endenden Verkehr.

%Vor dem großen Börsencrash wäre eine Wohnung wie diese unbezahlbar gewesen, doch als innerhalb von nur einem Wochenende fast die gesamte Mittelschicht des Westens in die Armut abrutschte, reagierten die Giga Corporations schnell und stellten ihre Miets- und Geschäftsmodelle um. Das neue System nannte sich Benefits Grid und ersetzt Löhne mit Subscriptions und Abos. Ob Wohnung, Fitnessstudio, Essen oder Videoportal, alles wurde von Corps bereitgestellt, wenn man nur tüchtig anpackte. Da in nur wenigen Tagen alle Verträge angepasst waren, munkelte manch einer, dass der Crash nur ein Vorwand für die Umstellung gewesen war. In der Flut an Nachrichten die in dieser Zeit um die Welt ging, ließ sich jedoch keine eindeutige Wahrheit mehr erkennen - jeder durfte glauben was er wollte.

%``Weißt du was mir noch Hoffnung macht Amir? Die ganzen Nachrichtenbots und Mainstream News, ja selbst die Alternativen Medien, erzählen alle irgendwelche Geschichten, aber eigentlich interessieren die schon lange niemanden mehr. Klar, jeder schaltet noch ab und zu seinen Lieblingskanal an, um ein letztes Gefühl von Kontrolle zu wahren, aber alles nur noch aus Gewohnheit. Tief im inneren weiß doch jeder, dass alles eine Farce ist. Keiner glaubt den Scheiß mehr! Und noch was: Keiner mag mehr an Fairness glauben...'' Harald machte eine kunstvolle Pause und nahm die Kippe aus seinem Mundwinkel, um demonstrative daran zu ziehen
%``..., aber jeder weiß noch was fair ist. Niemand sympathisiert mehr mit denen da oben.'' Harald lächelte verschwörerisch und nippte genüßlich an seinem Whiskey. 
