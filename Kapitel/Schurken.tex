\MosaicChapter{German}{Schurken}{Ein Paralleluniversum, eine Galaxis, einen Kontinent, ein Zeitalter, eine Gedankenwelt, aber vielleicht auch nur einen Brief entfernt:}{Ephraim Schott, Johannes Hartmann}
%\Chapter{Schurken}
\Authors{Ephraim Schott, Johannes Hartmann}
% \lettrine{}{} command typesets a drop cap in the document

% second brace set formats text placed inside as small caps
% (same thing as doing \lettrine{L}\textsc{orem ipsum ...}

\lettrine{``N}{un, sitzt ihr auch manchmal am Fenster} der Hafenkneipen des Nordens und blickt raus auf die

Eine Galaxis, einen Kontinent, ein Zeitalter, aber vielleicht auch nur einen Brief entfernt betrat das Mädchen den Laden und blickte sich um.

"Guten Tag", sagte der Otter am Tresen.
"Guten Tag", sagte das Mädchen.
Der Otter hüpfte von seinem Schemel und lief zweimal um das Mädchen herum.
"Du bist klein", sagte der Otter.
"Du auch", sagte das Mädchen.
Der Otter klettete zurück auf seinen Schemel.
"Jetzt bin ich nicht mehr klein."
"Das stimmt wohl", antwortete das Mädchen und zuckte mit den Schultern.
Das Mädchen blickte sich um, als wäre sie auf der Suche nach etwas, das sie auf den Regalen des kleinen Ladens finden könnte.
"Wonach suchst du?", fragte der Otter und setzte seine Brille auf.
"Nach einem Schurken."
"Einem Schurken?"
"Ja, einem Schurken."
"Nun... äh... wie dem auch sei, Schurken verkaufe ich hier nicht."
"Ich will doch garnichts kaufen", sagte das Mädchen trotzig, "dafür habe ich gar kein Geld, Herr Otter."
Herr Otter schüttelte den Kopf, sprang erneut von seinem Schemel, lief ein weiteres Mal um das Mädchen herum, schüttelte den Kopf erneut und kehrte zu seinem Schemel zurück. "Klein und schurkisch", sagte er, eher er sich wieder seiner Zeitung zuwand. 

Das Mädchen wanderte eine Weile scheinbar ziellos im Laden umher, ehe sie vor einem rötlichem Regal stehen blieb. 
"Hier," sagte sie, "Ich glaube hier ist er."