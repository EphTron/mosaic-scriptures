\MosaicChapter{German}{Treibholz}{Ein Paralleluniversum, eine Galaxis, einen Kontinent, ein Zeitalter, eine Gedankenwelt, aber vielleicht auch nur einen Brief entfernt:}{Ephraim Schott, Johannes Hartmann}
\Authors{Ephraim Schott, Johannes Hartmann}
% \lettrine{}{} command typesets a drop cap in the document

% second brace set formats text placed inside as small caps
% (same thing as doing \lettrine{L}\textsc{orem ipsum ...}

\lettrine{``N}{un, sitzt ihr auch manchmal am Fenster} der Hafenkneipen des Nordens und blickt raus auf die raue See, wie der Winter langsam herbeikommt? Die See selbst kann auch der Winter nicht zähmen, doch seht ihr aus dem Fenster und beobachtet wie sich langsam das Eis über die Docks zieht? Wie die Mäntel der Hafenhände Tag für Tag dicker werden, und die Glühweinbecher Tag für Tag schneller leer werden?''

``Seht ihr, ich kann euch auch nicht helfen. Nicht auf eurer Suche, nicht beim Finden. Es ist zu einfach, die Götter finden zu wollen, wenn die Sonne aufgeht. Man muss sie finden, allein, in der Kälte, im dunklen der Nacht.''

Der Fischer spuckte in den Spucknapf neben dem Eingang der Taverne und blickte einen jungen Mann, der gerade die Taverne betreten wollte, intensiv aus müden, ehrlichen Augen an, ehe er sich selbst in einer Wolke aus modrig riechendem Tabakrauch verhüllte.

Eigentlich war der flachsblonde, junge Mann nur am Hafen entlang spaziert. Als er von der Hafenpromenade aus, die Taverne mit den großen, von innen beschlagenen Fenstern und Blick aufs Wasser, erblickt hatte, war ihm diese, wie ein geeigneter Ort erschienen, um sich seine kalten Hände und Ohren aufzuwärmen. Zu tief in seinen Gedanken war er gewesen, um über die mögliche Gesellschaft des Ortes zu sinnieren.
Während er sich an dem stämmigen Fischer vorbeischob, holte ihn dessen raue Stimme jetzt um so schneller in die Gegenwart zurück.

Schüchtern nickend blieb der Jungspund zwischen Rauchschwaden und Eingangstür stehen und blickt zögerlich zwischen den großen, groben Stiefeln des Anderen und der Türschwelle hin und her. Unsicher was er auf den Monolog erwidern sollte, wurde er zum Gefangenen seiner eignen Höflichkeit, die von ihm verlangte dem Fremden zu antworten, bevor er ins Warme treten konnte, und seiner eigenen Schüchternheit, die es ihm verbot seine Gedanken für eine Antwort zu sammeln. Nach einer kurzen Pause, die dem Blonden wie eine elende Ewigkeit erschien, drehte er sich zum Fischer um und stammelte aufgeregt: ``Ja. Ich. Also, ich suche gar nichts. Äh. Ist drinnen noch Platz? Ich wollt' mich nur etwas auf-aufwärmen''. Ohne Blickkontakt mit dem Rauchenden aufzunehmen, zuckte er entschuldigend mit den Schultern, so als wollte er sagen ``Seht, ich kann euch auch nicht helfen''. Damit wandte er sich mit von der Kälte geröteten Backen der Tür zu und umgriff deren Knauf.

``Klar ist drinnen noch Platz. Aber was wollt ihr drinnen? Was erhofft ihr zu finden, wenn ihr doch nichts sucht?''

Der Fischer lächelt verschmitzt und wischte sich ein paar Schneeflocken von der Stirn, ehe er seine schmutzig-gelblich-braune Wollmütze wieder tiefer ins Gesicht zog.

Erneut spürte der junge Mann, wie die Natur der unbeantworteten Frage sich zu einem Sturm hinter ihm zusammen braute und begann ihn zurückzuziehen. Zu seinem Glück, lagen seine vor Kälte schmerzenden Finger aber schon auf dem hölzernen Griff der Tavernentür und gaben ihm Halt, so dass er sich aus dem Sog der sozialen Konversation befreien konnte. Mit einem Ruck öffnet er die Tür und Wärme und Stimmen drangen ihm entgegen. Schnell trat er ein und drehte sich auf der Türschwelle nochmals zum Fischer. 

``Ich wollte mich aufwärmen''.\\

Innen roch es nach Zedernholz und Tabak und ein kleines Feuer brannte in einem offenen Kamin. Die Spelunke, deren Namen er durch das Gerede des Fischers nicht einmal wahrgenommen hatte, war zu dieser Zeit des Tages noch nicht gut gefüllt. Ein paar Trunkenbolde saßen an ihren Stammplätzen, hinter der Bar arbeitete ein stämmiger Mann und eine hübsche Frau mit langen, eisblonden Haaren. Am Tresen selbst bei den Spucknäpfen saßen ein paar Pfeifenraucher, doch sonst konnte nicht die Rede von regen Treiben sein.

``Habt ihr Hunger?'', rief der Mann hinter dem Tresen, als er den Neuankömmling bemerkte. Der junge Mann nickte und trat näher an den Tresen, während er sich seine Jacke aufknöpfte. Kommentarlos lächelte ihm die Frau hinter dem Tresen zu, während der Wirt sich bereits daran machte eine Suppe aus einem gusseisernen Topf zu schöpfen. 
``Auch `was zu trinken, Großer?" raunte der Mann hinter dem Tresen und warf dem Jungspund dabei eine Blick über den Rand des Kessels zu. 
"Ja gern'. Ein warmes Ale. Port's Delight, wenn's recht ist.''

Der junge Mann ließ seinen Blick durch die Schenke gleiten. Entlang der großen, beschlagenen Fenster, von denen man auf das Treiben im Hafen und hinaus aufs Meer schauen konnte, standen einige kleine Tische mit je zwei Sitzplätzen. In der Nähe des warmen Kamins standen größere Tische mit dicken Kerzen an den je vier Personen Platz finden konnten. Die Wärme des offenen Kamins zog den blonden Mann an, doch er war alleine und es erschien ihm unhöflich, ja beinahe schon dreist, alleine einen der großen Tische zu besetzen. Deshalb setze er sich an eines der großen Fenster und blicke hinaus auf das Meer und die Docks, während er auf sein Essen wartete.

``Erst sagt ihr, ihr sucht nach nichts. Dann sagt ihr, ihr wollt euch aufwärmen, also sucht ihr nach Wärme. Und dann auch noch nach Bier. Glaubt ihr wirklich, dass ihr nach nichts sucht?'' murmelte aus dem Nichts eine Stimme zum jungen Mann, der sich nicht mal zur Seite drehen musste um zuordnen zu können, dass ihm der Fischer offenbar in die Spelunke gefolgt war.

Zögerlich wandte sich der Wartende dem Fischer zu, der sich an einen benachbarten Tisch gesetzt hatte und schaute ihm nun zum ersten Mal ins Gesicht. Eisblaue Augen blickten den junge Mann aus einem kantigen, hageren Gesicht entgegen. Die wettergegerbte Haut und kleine Falten um die tiefliegenden Augen und schmunzelnden Mundwinkel deuteten an, dass der Fischer schon weitaus mehr Winter erlebt hatte als er selbst, aber zu seiner Überraschung war der Fischer dennoch jünger als er es zuvor von der rauen Stimme her angenommen hatte. Auffällig waren die markanten Furchen im Gesicht des Fischers, die sich von den schlanken Nasenflügeln bis zu den Mundwinkeln zogen und von dem schlecht gestutztem Bart nicht versteckt werden konnten. Während er die Art des Fischers durchaus als aufdringlich Empfunden hatte, so strahlten die müden, blauen Augen eine innere Ruhe und ein wahres Interesse aus. Das schmunzeln des Fischers und die sich ausbreitende Wärme linderten die Unsicherheit des jungen Mannes etwas; doch die Frage baute sich vor ihm auf wie ein Richter und verlangte ein Rechtfertigung.
%setzte sich in seinen Nacken und verlangte ein Rechtfertigung.

``Treibholz'' brach es unbeholfen aus dem Blonden heraus.

Obwohl der Fischer sich direkt in den nächsten Schleier von Rauch verhüllte, schien es dem Jüngling dennoch so als ob die kalten, blauen Augen ihn noch immer fixieren würden. Für einen Moment fühlte er sich beobachtet, ja, fast schon verurteilt. Ohne genau nennen zu können woher das Gefühl kam, wanderte ein kalter Schauer seinen Rücken herab. Doch Zeit das Gefühl zu ergründen blieb nicht, den so schnell wie das Gefühl gekommen war, so schnell verschwand es wieder...

... was auch daran gelegen haben mag, dass der Wirt mit einem Knall ein Tablett mit einer dampfenden Schüssel Suppe und einem Humpen Ale neben dem Jüngling abstellte. ``Unsere Küstensud Suppe und ein Port's Delight''.

Der junge Mann blickte erleichtert auf und begann eifrig in seinen Manteltaschen nach Geld zu kramen.

``Zahlen könnt ihr später. Esst! So lange es noch heiß ist'' winkte der Wirt ab und begab sich zurück hinter den Tresen.

Der blonde Mann zog das Tablett näher an sich heran und blickte zum Fischer dessen Augen nun in die weite Ferne blickten, so als hätte der Jüngling ihn mit seiner Aussage an etwas erinnert, was er schon lang vergessen hatte. Der Fischer machte keine Anstalten etwas auf die Aussage des Jünglings zu erwidern, was diesem ganz recht war, denn die Fragen des Fischers bereiteten ihm Unbehagen.

Langsam begann der Jüngling die heiße Suppe zu essen und als es ihm gerade so schien als würde der Fremde auf seine Treibholz-Aussage nicht mehr reagieren, vernahm er ein tiefes Summen aus Richtung des Fischers, das zügig zu einer Art Gesang oder Murmeln anstieg.

\begin{itquote}
Die dunkle Welle, sie kam, sie ging.\\
Das Leben an Seil und Segel hing.\\
Durch schwarze Nacht trug Holz die Last.\\
Der Griff des Winters brach den Mast.\\
%Im dunklen leuchtet, ein greller Blitz, am Horizont ein letztes Licht.\\
Am Horizont leuchtet ein letztes Licht.\\
Doch jede Hoffnung nahm die Gischt.\\
Nicht Kompassnadel, nicht Seekart' und Sextant,\\
weder Stärke, Mut, noch tüchtig' Hand,\\
Nichts bricht den Sturm.\\

%Ein letztes Lied, erklingt nun hier.\\
%Auf dass der Vater der Wellen mir\\
%Mein Leiden nimmt und rette' mich.\\
%Auch Er leider nicht\\
%Doch mit etwas Tugend\\
%Öffnet Er nun sein Tor\\
%Für meine Seele\\
%Auf dass sie nicht ertrinke\\
%Im ewigen Nichts\\

Der Herr der Wellen steigt empor,\\
und legt mein Leib in kalte Ketten.\\
Tosend öffnet er sein Tor,\\
und zieht hinab in schwarze Betten.\\
Vor Kälte fast verbrennt der Leib,\\
gefangen bleibt mein stummes Klagen.\\
Wo Schreie war'n, nur Stille bleibt,\\
doch in der Brust will Leben schlagen.\\
Nichts bricht den Sturm.\\

So klammer ich,\\
Verzweifelt sterbend,\\
Am Treibholz mich fest.\\

So klammer ich,\\
Verzweifelt sterbend,\\
Am Treibholz mich fest.\\  
\end{itquote}


Die warme Suppe und die langsamen, wie Wellen rollenden Verse besänftigten das innere Tosen im Schädel des jungen Mannes. Zusätzlich hatte ihm der Fischer diesmal keine direkte Frage gestellte, was ihm Zeit gab sich zu sortieren, weil keine unmittelbare Antwort von ihm verlangt wurde.
Der junge Mann blickte den Fischer ruhig an und ließ die Zeilen einen Augenblick auf sich wirken.

``Hat die schiffbrüchige Person überlebt?'' fragt er mit ernster, gesenkter Stimme und neigte sich neugierig einen mü in Richtung des rauchenden Fischers.

``Was denkt ihr? Überlebt der Seemann, der sich bereits auf die Gedankenreise ins Reich des Vaters der Wellen macht?''

``Hmm...'', murmelte der junge Mann und löffelte verlegen in seiner Suppe, eher ein leises ``Ich weiß es nicht'' anfügte.

``Ich habe euch auch nicht gefragt ob ihr es wisst. Was denkt ihr?''

Der junge Mann blickte erschrocken auf. Etwas hatte sich in ihm geregt. Die Frage hatte ein unangenehmes Gefühl in ihm entfacht. Ein seltenes Gefühl, dass sich jetzt wie Säure in seinem Magen ausbreitete und schien ihm die Kehle hinauf zu steigen. Normalerweise richtet sich diese Gefühl gegen ihn selbst; doch dieses Mal galt es dem Fischer. Seine Finger umgriffen den Löffel fester, so dass seine Knöchel weiß hervortraten.

``Ihr wollt wissen was ich \textit{denke}?'' fragte er mit leiser, aber klarer Stimme und betonte das letzte Wort mit einer Schärfe, die durch die Rauchschwaden des Fischers schnitt. Der Klang von Unsicherheit war in den Worten des jungen Mannes nicht mehr zu finden und mit gepresster Stimme fuhr er schneller werdend fort:
``Wollt ihr wissen wie sich meine Gedanken im Kreis drehen. Wie ich über alles nachdenke. Wie ich alles überdenke? Denkt ihr eure Fragen machen es besser? Sehe ich vielleicht aus, als... als...''  der Jungspund hielt kurz inne um nach Luft und Worten zu schnappen und fuchtelte dabei verärgert mit dem Löffel in der Luft rum.

``...als würde ich mich über eure Fragen freuen? Ich sag euch was ich denke! Ich denke, dass ihr mich genug gefragt habt. Ich denke, dass ihr mir auf die Nerven geht! Ich denke, dass es unerhört ist, dass ihr mir hinter lauft und mir... mir...'' 

Der junge Mann rang wieder zornig nach Luft und den richtigen Worten, als er bemerkte, dass er mit seinem Löffel die Suppe über den gesamten Tisch verteilt hatte. Die Feststellung, dass er wegen dem Fischer jetzt auch noch den Tisch dreckig gemacht hatte, machte ihn noch wütender, unterbrach jedoch seinen Wortfluss.

Energisch griff er in seine Manteltasche und kramte ein altes Taschentuch hervor, mit dem er begann die Reste des Eintopfes vom Tisch zu wischen. Dabei blickte er wieder zum Fischer auf bereit ihm zu sagen, dass er ja wohl verrückt geworden sei. Doch der Mann saß nicht mehr an seinem Tisch. Der Jüngere blickte sich im Raum um, doch konnte den Fischer nirgends erblicken.\\

Verblüfft stand er auf und schaute sich verwundert in der Taverne um. Mit dem Fischer war auch sein Zorn verflogen.

``Entschuldigt, habt ihr zufällig den Fischer gesehen, der eben noch hier bei mir saß? Hat er die Spelunke verlassen?", fragte der Jüngling die Frau hinter dem Tresen, die mit dem Kopf schüttelte. 
``Nein, tut mir leid, ich muss verpasst haben dass sich jemand zu euch gesetzt hatte.''
``Und ihr?'', fragte der junge Mann den Herrn am Tresen, doch er bekam nur ein Grunzen als Antwort, das er als Nein deutete.

Ungläubig blickte der junge Mann zurück zu dem Platz an dem der Fischer gesessen hatte. Wurde ihm ein übler Streich gespielt? War das ein Scherz den sich die Hafenarbeiter hier mit Unbekannten in dieser Taverne machten?
Sein Blick wanderte zu den anderen Spielern und Trinkern. Ihre Aufmerksamkeit galt den Karten und den Humpen. Niemand schielte verstohlen oder verhöhnende zu ihm rüber.

Wie konnte die hübsche Frau den Fischer nicht gesehen haben? 
``Er hatte eine gelb-braune Wollmü-'' began der Flachsblonde seine Frage, während er sich am Hinterkopf kratzte; doch er verstummte mitten im Satz, als seine Finger den Garn einer weichen Mütze berührten.
Der Jungspund began verlegen in seiner Tasche zu kramen und wäre am liebsten in ihr verschwunden. Bestimmt schaute ihn die Frau hinter der Bar an wie einen Verrückten. Er konnte ihren Blick förmlich auf seiner Haut brennen spüren und seine Backen begannen zu glühen.
Hatte er sich den Fischer eingebildet? War das möglich? 
Für den Bruchteil einer Sekunde konnte er die tiefen eisblauen Augen vor sich sehen.
Er konnte die Fragen, die ihm der Fischer gestellt hatte noch in seiner Brust brennen spüren, ähnlich wie die Blicke der Barfrau.\\

Was hatte ihn an die Tür dieser Taverne geführt? Hatte er tatsächlich etwas gesucht? Hatte er Kurs gesetzt und war entschieden mit vollen Segeln hier eingelaufen, um die Spelunke zu betreten? Oder war es die Kälte gewesen, die ihn wie Treibholz, das von den Wellen an fremde Ufer getragen wird, an die Türschwelle der Spelunke geschwemmt hatte?

Er wandte sein Blick zu den Fenstern und schaute hinaus aufs Meer.
War er verrückt geworden?
Er konnte keine Antwort darauf finden.\\
%Eine Ruhe breitete sich in ihm aus.

Draußen hatte es angefangen zu schneien und große Schneeflocken tanzten vom Himmel herab und fielen ins schwarze Meer, wo sie für immer verschwanden und eins wurden mit dem dunklen Salzwasser.

